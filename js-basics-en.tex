\documentclass[10pt,a4paper]{article}
\usepackage[english]{babel}

\usepackage[utf8]{inputenc}
\usepackage[german]{babel}

\usepackage[landscape,margin=1cm]{geometry}
\usepackage[default]{raleway}
\usepackage{multicol}

\usepackage{inconsolata}
\usepackage[T1]{fontenc}
\usepackage{fontawesome}

\usepackage{xcolor}
\usepackage[hidelinks]{hyperref}
\usepackage[inkscapepath=build/svg-latex]{svg}

\setlength{\columnsep}{1cm}
\setlength{\parindent}{0pt}
\pagestyle{empty}

\usepackage[
    type={CC},
    modifier={by-nc-sa},
    version={4.0},
]{doclicense}


\newcommand{\attribution}[1]{
\interlinepenalty 10000
\vspace{\baselineskip}
\linethickness{0.4mm} % Thickness of the footer line
{\color{accentcolor}\line(1,0){45}} % Print the line with a custom color
\footnotesize{
\begin{tabbing}
\faicon{users}~~\= Created by #1\\
\faicon{github} \> Source: \url{https://github.com/ccoenen/cheatsheets}\\
\faicon{calendar} \> Date: \today\\
\end{tabbing}
\doclicenseThis
}
}

\usepackage[outputdir=build]{minted}
\usepackage{tcolorbox}
\tcbuselibrary{most,listings,minted}

\tcbset{
tcbox width=auto,
left=0mm,
top=0mm,
bottom=0mm,
right=0mm,
boxsep=1mm,
middle=1mm
}

\newtcblisting{codebox}[2]{
colback=black!7,
colframe=black!0,
colbacktitle=black!30,
top=0mm,
toptitle=0mm, 
left=0mm,
bottom=0mm,
arc=0pt,
outer arc=0pt,
attach boxed title to top right={xshift=1mm,yshift=-4mm},
title=#1,
fonttitle=\color{white},
enhanced,
minted options={
  numbers=left,
  style=friendly,
  fontsize=\scriptsize,
  breaklines,
  breaksymbolleft=,
  autogobble,
  linenos=false,
  numbersep=0.7em
},
minted language=#1,
listing only, 
listing engine=minted,
}

\usepackage[
style=alphabetic,
sorting=nyt,
autolang=hyphen, 		
backend=biber
]{biblatex}

\AtNextBibliography{\footnotesize}


\definecolor{accentcolor}{HTML}{cad804}
\bibliography{js-basics}

\begin{document}

{\color{accentcolor}JavaScript Basics Cheatsheet}

\begin{multicols}{3}

\scriptsize

% \textbf{} <-- strong
% \texttt{} <-- monospaced, no hightlighting
% \enquote{} <-- put stuff in correct quotes
% \begin{codebox}{lang}{} <-- start a codebox in "lang" language
% \end{codebox} <-- finishes the codeblock started with \begin{codebox}

%--------------MAIN CONTENT HERE----------------

JavaScript is not Java - it's not even related to Java. It has existed since 1997 and was originally created as a scripting language for browsers. A great source of reliable information is the JavaScript reference on the Mozilla Developer Network \cite{mdn-js}.

\section*{Building Blocks}

\subsection*{Variables}
Variables contain values. Variables are created (\enquote{declared}) with \texttt{let}. There are two other similar things: \texttt{const} creates variables that can't be re-assigned and \texttt{var} is a legacy way of declaring variables. Following the declaration usually comes the name of the variable, an equals sign (\enquote{=}) and some value that is assigned to the variable.

\begin{codebox}{js}{}
  let greeting = "Hello World";
  const ANSWER = 42;
\end{codebox}

\subsection*{Functions}
Functions create \textbf{reusable} code. You can run them as often as you want. They are created with the keyword \texttt{function}, followed by the function name, parentheses and curly braces. The parentheses can contain parameters, the curly braces contain the instructions the function should execute.

\begin{codebox}{js}{}
  function add(a, b) {
    // instructions how to add two numbers
  }
\end{codebox}
We can execute (or \enquote{call}) our function like this:
\begin{codebox}{js}{}
  add(5, 7);
\end{codebox}
\texttt{5} and \texttt{7} are \textbf{parameters} of the function that can be used within the function's instructions.

Functions can also \textbf{return} values (give values back to the place calling the function). To do this, we use the keyword \texttt{return} followed by the value your function should return.

\begin{codebox}{js}{}
  function add(a, b) {
    let sum = a + b;
    return sum;
  }
  
  let lifeChangingResult = add(17, 25);
  console.log(lifeChangingResult); // prints 42.
\end{codebox}

There's also a shorthand for functions that is very often seen in event handlers: \texttt{() => {}}. We call this an \textbf{arrow-function}.
\begin{codebox}{js}{}
  button.addEventListener("click", (event) => {
    event.preventDefault();
  });
\end{codebox}
Here, we create an anonymous arrow function that takes a parameter \texttt{event}. We add this as an eventListener that reacts to a mouse click on some button element that's not part of the example.

\subsection*{Comments}
Comments can help by adding descriptions to parts of your code. Comments are \textbf{not executed} by the computer. JavaScript knows single line comments that start with \texttt{//} and continue to the end of the line:
\begin{codebox}{js}{}
  // This is a neat comment!
  let city = "Arendelle"; // this also works "after" code
\end{codebox}

JavaScript also has multi-line comments which span from \texttt{/*} to \texttt{*/}.
\begin{codebox}{js}{}
  /* within these, you can have a
     larger amount of text.
     Just like this. */
\end{codebox}

It is a common technique to use comments to \enquote{comment out} code that you don't want to run. It's useful for debugging.
\begin{codebox}{js}{}
  // let powerLevel = 42; // this was too low.
  let powerLevel = 9001; // IT'S OVER 9000!

  /*
  let x = 100;
  let y = 200;
  let radius = 20;
  */
  let data = {
    x: 100,
    y: 200,
    radius: 20
  };
\end{codebox}

\subsection*{Strings}
A String can contain text and similar things.
\begin{codebox}{js}{}
  let name = "Harald Töpfer";
\end{codebox}
It is possible to combine Strings:
\begin{codebox}{js}{}
  let firstName = "Harald";
  let lastName = "Töpfer";
  let name = firstName + " " + lastName;
\end{codebox}

\subsection*{Numbers}
JavaScript has a single type for numbers, called \texttt{Number}. Internally this is a IEEE-754 "double" (64 bit) value. Numbers are not written within quotes (or they would become Strings).

\begin{codebox}{js}{}
  let myInteger = 42;
  let myFloat = 13.5;
\end{codebox}
It's possible to make calculations with variables of the type \texttt{Number}:
\begin{codebox}{js}{}
  let rows = 25;
  let columns = 3;
  let fields = rows * columns; // "fields" will be 75
\end{codebox}

\subsection*{Booleans / Bool}
A \texttt{Bool} contains a boolean value. In other words: a value that can only be \texttt{true} or \texttt{false}.
\begin{codebox}{js}{}
  let darkMode = true;
  let dyslexicFont = false;
\end{codebox}


\section*{Output}
\texttt{console.log()} can be used to output something, for example while you are debugging:
\begin{codebox}{js}{}
let myInteger = 42;
console.log(myInteger); // will output 42
\end{codebox}
This output will appear in your browser's developer tools if you are creating a website, or in your terminal if you are creating a terminal application.


\section*{Conditions and Branching}
Code in curly braces will only be run if a condition in the parentheses is met.

\begin{codebox}{js}{}
  let x = 100;
  if (x === 100) {
    // This right here will only be executed
    // if x is exactly 100
  }
\end{codebox}
Please note that in JavaScript, comparing two values should usually be done with type safe comparison \texttt{===}.

Your program might need to branch out in more different possibilities, and that's where \texttt{else if} and \texttt{else} come in. In a construction like that only one of the blocks will be executed.
\begin{codebox}{js}{}
  let x = 100;
  if (x < 100) {
    // Will be executed if x is smaller than 100
  } else if (x < 200) {
    // will be executed if x is greater than 100
    // but also smaller than 200
  } else {
    // will be run in any other case
  }
\end{codebox}


\section*{Operators}
\subsection*{Comparisons}
Most often, you will find comparison operators in an if-statement like this:

\begin{codebox}{js}{}
  if (x < 100) {
    // Only execute the code if
    // x is less than 100.
  }
\end{codebox}
If a comparison checks out, it returns \texttt{true}, otherwise it returns \texttt{false}. Here's a selection of common comparison operators:

\vspace{0.5cm}
\begin{tabular}{c l}
  \texttt{===} & checks for \textbf{strict equality} of left and right value \\
  \texttt{!==} & checks for \textbf{unequality} of left and right value \\
  \texttt{>} & checks if left is \textbf{greater than} right \\
  \texttt{>=} & checks if left is \textbf{greater than or equal to} right \\
  \texttt{<} & checks if left is \textbf{smaller than} right \\
  \texttt{<=} & checks if left is \textbf{smaller than or equal to} right \\
\end{tabular}

Additionally, there are simple comparisons \texttt{==} and \texttt{!=} which compare across data types. For these, \texttt{5 == "5"} would be \texttt{true}. Usually, you should stick to the type safe comparisons \texttt{===} and \texttt{!==}.

\subsection*{Logical Combinations}
These operators let you combine logical expressions.

The \textbf{AND}-operator \texttt{\&\&} requires both conditions to be \texttt{true}:
\begin{codebox}{js}{}
  if (player.allergicTo === "nuts" && food === "nut") {
    // when both things are true, we execute these lines.
    player.health--;
  }
\end{codebox}

The \textbf{OR}-operator \texttt{||} requires any condition to be \texttt{true}:
\begin{codebox}{js}{}
  if (waitingTime > 10 || lectureDuration > 120) {
    getBored();
  }
\end{codebox}
These expressions may be combined and can get as complex as you need them to be. For better readability, it is sometimes helpful to use parentheses.

\section*{Helpful Things}
\subsection*{Rounding Numbers}
Standard rounding of numbers (values below 0.5 will be rounded down, starting from 0.5 they will be rounded up) can be done like this:

\begin{codebox}{js}{}
  let value = Math.round(12.3); // value will be 12
\end{codebox}
By contrast, \texttt{floor} means going \enquote{down} to the previous integer
\begin{codebox}{js}{}
  let value = Math.floor(12.9999999); // value will be 12
\end{codebox}
And finally \texttt{ceil} (\enquote{ceiling}) means going \enquote{up} the the next integer
\begin{codebox}{js}{}
  let value = Math.ceil(12.3); // value will be 13
\end{codebox}

%\subsection*{Zufall}
%\subsection*{Winkel}
%\subsection*{Datum und Zeit}

%-------------MAIN CONTENT DONE----------------


\printbibliography
\attribution{Garrit Schaap \& Claudius Coenen}
\end{multicols}
\end{document}
