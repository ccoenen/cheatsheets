\documentclass[10pt,a4paper]{article}

\usepackage[utf8]{inputenc}
\usepackage[german]{babel}

\usepackage[landscape,margin=1cm]{geometry}
\usepackage[default]{raleway}
\usepackage{inconsolata}
\usepackage[T1]{fontenc}
\usepackage{fontawesome}

\usepackage{multicol}
\setlength{\columnsep}{1cm}

\setlength{\parindent}{0pt}

\usepackage{xcolor}

\usepackage[
    type={CC},
    modifier={by-nc-sa},
    version={4.0},
]{doclicense}


\newcommand{\attribution}[1]{
\interlinepenalty 10000
\vspace{\baselineskip}
\linethickness{0.4mm} % Thickness of the footer line
{\color{accentcolor}\line(1,0){45}} % Print the line with a custom color
\footnotesize{
\begin{tabbing}
\faicon{users}~~\= Created by #1\\
\faicon{github} \> Source: \url{https://github.com/ccoenen/cheatsheets}\\
\faicon{calendar} \> Date: \today\\
\end{tabbing}
\doclicenseThis
}
}

\usepackage[outputdir=build]{minted}
\usepackage{tcolorbox}
\tcbuselibrary{most,listings,minted}

\tcbset{
tcbox width=auto,
left=0mm,
top=0mm,
bottom=0mm,
right=0mm,
boxsep=1mm,
middle=1mm
}

\newtcblisting{codebox}[2]{
colback=black!7,
colframe=black!0,
colbacktitle=black!30,
top=0mm,
toptitle=0mm, 
left=0mm,
bottom=0mm,
arc=0pt,
outer arc=0pt,
attach boxed title to top right={xshift=1mm,yshift=-4mm},
title=#1,
fonttitle=\color{white},
enhanced,
minted options={
  numbers=left,
  style=friendly,
  fontsize=\scriptsize,
  breaklines,
  breaksymbolleft=,
  autogobble,
  linenos=false,
  numbersep=0.7em
},
minted language=#1,
listing only, 
listing engine=minted,
}

\usepackage[
style=alphabetic,
sorting=nyt,
autolang=hyphen, 		
backend=biber
]{biblatex}

\AtNextBibliography{\footnotesize}


\definecolor{accentcolor}{HTML}{007700}
\bibliography{css-basics-literature}

\begin{document}

{\color{accentcolor}CSS Basics Cheatsheet}

\begin{multicols}{3}

\scriptsize
%--------------MAIN CONTENT HERE----------------

\section*{Einbinden}
CSS wird in HTML-Dokumenten verwendet, um Elemente zu stylen. Es gibt drei Möglichkeiten, CSS zu verwenden:

\subsection*{CSS-Datei}
Eine eigenständige Datei kann in HTML mit dem Link-Tag eingebunden werden.
\begin{codebox}{html}{}
<link rel="stylesheet" href="some-file-name.css">
\end{codebox}
Dieses Link-Tag steht üblicherweise im Head-Element in HTML.

\subsection*{Style-Tag}
Im Style-Tag stehen die CSS-Definitionen direkt in der HTML-Datei.
\begin{codebox}{html}{}
<style>
/* Ihr CSS hier! */
</style>
\end{codebox}
Das Style-Tag steht ebenfalls oft im Head-Element.

\subsection*{Style-Attribut}
HTML-Elemente können ein style-Attribut haben, in dem direkt CSS-Eigenschaften stehen.
\begin{codebox}{html}{}
Ein <span style="color: black">schwarzes</span> Wort.
\end{codebox}

\section*{Selektoren}
Selektoren bestimmen, was einen Style erhalten soll. Hiermit wählen Sie HTML-Elemente, die verändert werden sollen.
\vspace{0.2cm}

Der \textbf{Element-Selektor} wählt alle Elemente direkt über den Namen des Elements aus.
\begin{codebox}{html}{}
<h1>Meine Überschrift</h1>
\end{codebox}
\begin{codebox}{css}{}
h1 {
  font-size: 10em;
}
\end{codebox}

Der \textbf{Class-Selektor} kann über das class-Attribut eines Elements gesetzt werden.
\begin{codebox}{html}{}
<div class="content">Inhalt</div>
\end{codebox}
\begin{codebox}{css}{}
.content {
  color: #FF00FF;
}
\end{codebox}
Man kann auch mehrere CSS Klassen, getrennt durch Leerzeichen, auf ein Element anwenden.
\begin{codebox}{html}{}
<div class="content blue">Inhalt</div>
\end{codebox}
\begin{codebox}{css}{}
.content {
  color: #FF00FF;
}
.blue {
  background-color: #0000ff;
}
\end{codebox}

Der \textbf{id-Selektor} wählt das Element mit der zugehörigen id aus. Achtung eine id darf nur einmal pro Dokument vergeben werden.
\begin{codebox}{html}{}
<div id="name">Garrit</div>
\end{codebox}
\begin{codebox}{css}{}
#name {
  font-size: 3em;
  color: #00ff00;
}
\end{codebox}


\section*{Einheiten}
\subsection*{Farben}
\begin{codebox}{css}{}
color: red;            /* Schlüsselwort */
color: #00ff00         /* Hex-Farbcode */
color: rgb(0, 0, 255); /* rgb Schreibweise */
\end{codebox}

\subsection*{Längen und Abstände}
\begin{codebox}{css}{}
padding: 0;       /* 0 braucht keine Einheit. */
font-size: 12pt;  /* Punkt */
border: 1px       /* Pixel */
line-height: 1.2em; /* Größe relativ zur Breite eines "m" */
\end{codebox}

Achtung: Pixel sind gedachte 96dpi Pixel, und damit oftmals nicht wirklich exakt ein Pixel.


\section*{Eigenschaften}
Eigenschaften müssen in Selektoren stehen. Es gibt buchstäblich hunderte CSS-Eigenschaften\cite{w3c-css}, hier daher nur eine kleine Auswahl.

\subsection*{Aussehen}
Die \textbf{background-color}-Eigenschaft setzt die Hintergrundfarbe eines Elements.
\begin{codebox}{css}{title="test"}
background-color: #FF0000;
\end{codebox}

Die \textbf{background-image}-Eigenschaft legt ein Hintergrundbild für ein Element fest. Hintergrundbilder werden immer über Hintergrundfarben gelegt.
\begin{codebox}{css}{}
background-image: url("pfad/zum/bild.jpg");
\end{codebox}

Die \textbf{border}-Eigenschaft legt die Eigenschaften des Rahmens eines Elementes fest
\begin{codebox}{css}{}
border-width: 1px;
border-style: solid;
border-color: #00FF00;
\end{codebox}



\subsection*{Layout}
Die \textbf{float}-Eigenschaft bestimmt, ob ein Element aus dem normalen Fluss herausgelöst werden soll und nachfolgende Inhalte auf der linken oder rechten Seite des Elements platziert werden sollen.
\begin{codebox}{css}{}
float: left;
float: right;
float: none;
\end{codebox}

Die \textbf{margin}-Eigenschaften sind für die Außenabstände aller vier Seiten eines Elements verantwortlich.
\begin{codebox}{css}{}
margin-top: 10px;
margin-right: 5%;
margin-bottom: 1em;
margin-left: auto;
\end{codebox}

Die \textbf{padding}-Eigenschaften sind für die Innenabstände aller vier Seiten eines Elementes verantwortlich.
\begin{codebox}{css}{}
padding-top: 10px;
padding-right: 5%;
padding-bottom: 1em;
padding-left: auto;
\end{codebox}

Die \textbf{position}-Eigenschaft legt die Positionsart eines Elements fest. Für die Positionierung selbst werden die Eigenschaften top, right, bottom oder left verwendet.
\begin{codebox}{css}{}
position: absolute;
position: relative;
position: fixed;
\end{codebox}

% TODO: Box-Modell-Grafik irgendwo?

\subsection*{Text}

Die CSS Eigenschaft \textbf{color} setzt die Vordergrundfarbe des Textinhalts eines Elements und seiner Dekorationen.
\begin{codebox}{css}{}
color: #FF00FF;
\end{codebox}

Die CSS Eigenschaft \textbf{text-align} beschreibt, wie Inlineinhalte wie Text in ihrem Elternblockelement ausgerichtet werden.
\begin{codebox}{css}{}
text-align: center;
\end{codebox}

Die \textbf{text-decoration} Eigenschaft wird dazu verwendet, die Textformatierung auf underline, overline, line-through oder blink zu setzen.
\begin{codebox}{css}{}
text-decoration: underline;
\end{codebox}

Die \textbf{font-family} erlaubt es, eine priorisierte Liste von Schriftfamiliennamen für ein ausgewähltes Element anzugeben. Achtung: Die Schrift muss auf dem Anzeigegerät auch vorhanden sein!
\begin{codebox}{css}{}
font-family: "Helvetica", "Arial", sans-serif;
\end{codebox}

Die \textbf{font-size} spezifiziert die Schriftgöße.
\begin{codebox}{css}{}
font-size: 12px;
\end{codebox}

Die \textbf{font-style} legt fest ob eine Schriftart als normal, italic oder oblique angezeigt wird.
\begin{codebox}{css}{}
font-style: italic;
\end{codebox}

Die \textbf{font-weight} definiert die Dicke der Schrift. Einige Schriftarten sind jedoch nicht in allen Werten verfügbar und unterstützen bspw. nur die Werte normal und bold.
\begin{codebox}{css}{}
font-weight: bold;
\end{codebox}


%-------------MAIN CONTENT DONE----------------


\printbibliography
\attribution{Garrit Schaap \& Claudius Coenen}
\end{multicols}
\end{document}
