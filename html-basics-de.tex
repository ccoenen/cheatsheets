\documentclass[10pt,a4paper]{article}
\usepackage[ngerman]{babel}

\usepackage[utf8]{inputenc}
\usepackage[german]{babel}

\usepackage[landscape,margin=1cm]{geometry}
\usepackage[default]{raleway}
\usepackage{inconsolata}
\usepackage[T1]{fontenc}
\usepackage{fontawesome}

\usepackage{multicol}
\setlength{\columnsep}{1cm}

\setlength{\parindent}{0pt}

\usepackage{xcolor}

\usepackage[
    type={CC},
    modifier={by-nc-sa},
    version={4.0},
]{doclicense}


\newcommand{\attribution}[1]{
\interlinepenalty 10000
\vspace{\baselineskip}
\linethickness{0.4mm} % Thickness of the footer line
{\color{accentcolor}\line(1,0){45}} % Print the line with a custom color
\footnotesize{
\begin{tabbing}
\faicon{users}~~\= Created by #1\\
\faicon{github} \> Source: \url{https://github.com/ccoenen/cheatsheets}\\
\faicon{calendar} \> Date: \today\\
\end{tabbing}
\doclicenseThis
}
}

\usepackage[outputdir=build]{minted}
\usepackage{tcolorbox}
\tcbuselibrary{most,listings,minted}

\tcbset{
tcbox width=auto,
left=0mm,
top=0mm,
bottom=0mm,
right=0mm,
boxsep=1mm,
middle=1mm
}

\newtcblisting{codebox}[2]{
colback=black!7,
colframe=black!0,
colbacktitle=black!30,
top=0mm,
toptitle=0mm, 
left=0mm,
bottom=0mm,
arc=0pt,
outer arc=0pt,
attach boxed title to top right={xshift=1mm,yshift=-4mm},
title=#1,
fonttitle=\color{white},
enhanced,
minted options={
  numbers=left,
  style=friendly,
  fontsize=\scriptsize,
  breaklines,
  breaksymbolleft=,
  autogobble,
  linenos=false,
  numbersep=0.7em
},
minted language=#1,
listing only, 
listing engine=minted,
}

\usepackage[
style=alphabetic,
sorting=nyt,
autolang=hyphen, 		
backend=biber
]{biblatex}

\AtNextBibliography{\footnotesize}


\definecolor{accentcolor}{HTML}{0000CC}
\bibliography{html-basics}

\begin{document}

{\color{accentcolor}HTML Basics Cheatsheet}

\begin{multicols}{3}

\scriptsize
%--------------MAIN CONTENT HERE----------------

\textbf{H}yper\textbf{t}ext \textbf{M}arkup \textbf{L}anguage

Mit dieser Seitenbeschreibungssprache werden Internetseiten und andere Dokumente geschrieben. Die Aufteilung ist \textbf{semantisch}, wir geben den Computern damit zu verstehen, welcher Teil des Dokuments welche Aufgabe hat. HTML kommt in der Regel gemeinsam mit CSS (Styling) und JavaScript (Verhalten) zum Einsatz.

Gute weiteführende Quellen sind SELFHTML \cite{selfhtml} und das Mozilla Developer Network \cite{mdn}.

\section*{Aufbau}

\includesvg[width=\columnwidth]{html-tag-parts.svg}

HTML besteht aus \textbf{Tags}, die Text oder andere Tags enthalten können:
\begin{codebox}{html}{}
<title>HTML Basics Cheatsheet</title>
\end{codebox}

Manche Tags haben auch \textbf{Attribute}:
\begin{codebox}{html}{}
<a href="portfolio.html">Oh, mein Portfolio!</a>
\end{codebox}

In HTML gibt es außerdem \textbf{Kommentare}:
\begin{codebox}{html}{}
<!-- hier in der Mitte ist der Kommentar -->
\end{codebox}


\subsection*{Grundgerüst}
Einige Dinge muss jedes HTML-Dokument enthalten, das Folgende ist das Minimale, was eine HTML-Datei braucht:
\begin{codebox}{html}{}
<!DOCTYPE html>
<html lang="de">
  <head>
    <meta charset="utf-8">
    <meta name="viewport" content="width=device-width, initial-scale=1.0">
    <title>Titel</title>
  </head>
  <body>
    <!-- hier folgt das eigentliche Dokument -->
  </body>
</html>
\end{codebox}


\section*{Tags}
\subsection*{Grundstruktur und Metainformationen}
Das \mintinline{html}|<html>|-Tag ist der Wurzelknoten eines HTML-Dokuments. Alle weiteren Elemente müssen in diesem Element enthalten sein. (,,Nachkommen`` bzw. Kind-Elemente).
\begin{codebox}{html}{}
<html> ... </html>
\end{codebox}

Im \mintinline{html}|<head>| werden Metadaten des Dokuments gesammelt. Hierzu gehören auch Skripte und Stylesheets.
\begin{codebox}{html}{}
<head> ... </head>
\end{codebox}

Im \mintinline{html}|<body>| steht der Hauptinhalt eines HTML-Dokuments. Jedes Dokument muss genau ein \mintinline{html}|<body>|-Element besitzen.
\begin{codebox}{html}{}
<body> ... </body>
\end{codebox}

\mintinline{html}|<title>| definiert Titel eines Dokuments, der in der Titelzeile des Browsers oder im Tab der betreffenden Seite angezeigt wird. Darf ausschließlich Text enthalten. Eventuell enthaltene Tags werden nicht interpretiert.
\begin{codebox}{html}{}
<title>Meine Webseite</title>
\end{codebox}

\mintinline{html}|<meta>| Wird für die Definition von Metadaten verwenden, die mit keinem anderen HTML-Element definiert werden können. Zum Beispiel welches Charakter-Set man verwendet. Meta darf selbst keine Tags enthalten. Es hat auch kein schließendes Tag.
\begin{codebox}{html}{}
<meta charset="utf-8">
\end{codebox}

\mintinline{html}|<link>| Wird verwendet, um externe CSS-Dateien in das aktuelle HTML-Dokument einzubinden, oder auch um auf alternative Versionen des selben Dokuments zu verlinken (z.B. RSS Feeds):
\begin{codebox}{html}{}
<link rel="stylesheet" type="text/css" href="example.css">
<link rel="icon" href="/favicon.ico">
<link rel="alternate" type="application/rss+xml" href="https://example.com/feed/">
\end{codebox}

\mintinline{html}|<style>| kann Definition CSS-Definitionen direkt enthalten:
\begin{codebox}{html}{}
<style>
h1 {
  margin-left: 2em;
}
</style>
\end{codebox}

\mintinline{html}|<script>| kann entweder auf Scripte verweisen oder direkt JavaScript-Code enthalten:

\begin{codebox}{html}{}
<script src="javascript-datei.js">
\end{codebox}

\begin{codebox}{html}{}
<script>
console.log("Hallo, Welt");
</script>
\end{codebox}

Die Script-Tags werden direkt an der Stelle ausgeführt, wo sie stehen. Üblicherweise packt man Sie ans Ende des Dokuments vor das \mintinline{html}|</body>|-Tag oder in den \mintinline{html}|<head>|.


\subsection*{Text-Auszeichnung}

Der Inhalt dieses Elements soll als Absatz dargestellt werden.
\begin{codebox}{html}{}
<p></p>
\end{codebox}


Hiermit werden Überschriften definiert. Es gibt sechs verschiedene Hierarchieebenen, wobei <h1> für die Hauptüberschrift steht und <h6> für eine Überschrift der untersten Ebene.
\begin{codebox}{html}{}
<h1></h1>
<h2></h2>
...
<h6></h6>
\end{codebox}

Bezeichnet einen Zeilenumbruch.
\begin{codebox}{html}{}
<br>
\end{codebox}

Bezeichnet einen thematischen Bruch zwischen Absätzen eines Abschnitts, Artikels oder anderem längeren Inhalt.
\begin{codebox}{html}{}
<hr>
\end{codebox}

Listen
\begin{codebox}{html}{}
<ol>
  <li>Eintrag Nr. 1</li>
  <li>Eintrag Nr. 2</li>
</ol>
<ul>
  <li>Wurst</li>
  <li>Käse</li>
</ul>
\end{codebox}



\subsection*{Layout}

Bezeichnet ein allgemeines Container-Element ohne spezielle semantische Bedeutung.
\begin{codebox}{html}{}
<div>
\end{codebox}

Markiert einen allgemeinen Textabschnitt.
\begin{codebox}{html}{}
<span>
\end{codebox}

Definiert den Kopfteil ("header") einer Seite oder eines Abschnitts. Er enthält oft ein Logo, den Titel der Website und die Seitennavigation.
\begin{codebox}{html}{}
<header>
\end{codebox}

Beschreibt einen Abschnitt der ausschließlich Navigationslinks enthält.
\begin{codebox}{html}{}
<nav>
\end{codebox}

Definiert den Hauptinhalt der Seite. Es ist nur ein <main> Element pro Seite zulässig.
\begin{codebox}{html}{}
<main>
\end{codebox}

Beschreibt einen Abschnitt eines Dokuments.
\begin{codebox}{html}{}
<section>
\end{codebox}

Beschreibt eigenständigen Inhalt, der unabhängig von den übrigen Inhalten sein kann.
\begin{codebox}{html}{}
<article>
\end{codebox}

Steht für eine Randbemerkung. Der übrige Inhalt sollte auch verständlich sein, wenn dieses Element entfernt wird.
\begin{codebox}{html}{}
<aside>
\end{codebox}

Definiert den Fußteil ("footer") einer Seite oder eines Abschnitts. Er enthält oft Copyright-Hinweise, einen Link auf das Impressum oder Kontaktadressen.
\begin{codebox}{html}{}
<footer>
\end{codebox}


%-------------MAIN CONTENT DONE----------------


\printbibliography
\attribution{Garrit Schaap \& Claudius Coenen}
\end{multicols}
\end{document}
