\documentclass[10pt,a4paper]{article}
\usepackage[ngerman]{babel}

\usepackage[utf8]{inputenc}
\usepackage[german]{babel}

\usepackage[landscape,margin=1cm]{geometry}
\usepackage[default]{raleway}
\usepackage{inconsolata}
\usepackage[T1]{fontenc}
\usepackage{fontawesome}

\usepackage{multicol}
\setlength{\columnsep}{1cm}

\setlength{\parindent}{0pt}

\usepackage{xcolor}

\usepackage[
    type={CC},
    modifier={by-nc-sa},
    version={4.0},
]{doclicense}


\newcommand{\attribution}[1]{
\interlinepenalty 10000
\vspace{\baselineskip}
\linethickness{0.4mm} % Thickness of the footer line
{\color{accentcolor}\line(1,0){45}} % Print the line with a custom color
\footnotesize{
\begin{tabbing}
\faicon{users}~~\= Created by #1\\
\faicon{github} \> Source: \url{https://github.com/ccoenen/cheatsheets}\\
\faicon{calendar} \> Date: \today\\
\end{tabbing}
\doclicenseThis
}
}

\usepackage[outputdir=build]{minted}
\usepackage{tcolorbox}
\tcbuselibrary{most,listings,minted}

\tcbset{
tcbox width=auto,
left=0mm,
top=0mm,
bottom=0mm,
right=0mm,
boxsep=1mm,
middle=1mm
}

\newtcblisting{codebox}[2]{
colback=black!7,
colframe=black!0,
colbacktitle=black!30,
top=0mm,
toptitle=0mm, 
left=0mm,
bottom=0mm,
arc=0pt,
outer arc=0pt,
attach boxed title to top right={xshift=1mm,yshift=-4mm},
title=#1,
fonttitle=\color{white},
enhanced,
minted options={
  numbers=left,
  style=friendly,
  fontsize=\scriptsize,
  breaklines,
  breaksymbolleft=,
  autogobble,
  linenos=false,
  numbersep=0.7em
},
minted language=#1,
listing only, 
listing engine=minted,
}

\usepackage[
style=alphabetic,
sorting=nyt,
autolang=hyphen, 		
backend=biber
]{biblatex}

\AtNextBibliography{\footnotesize}


\definecolor{accentcolor}{HTML}{dfc200}
\bibliography{js-oop-es6-de}

\begin{document}

{\color{accentcolor}Objektorientierung und Module in JavaScript}

\begin{multicols}{3}

\scriptsize
%--------------MAIN CONTENT HERE----------------

Objektorientiertes Programmieren ist eine Art zu programmieren. Hierbei ist ein großer Teil des Programmcodes in Objekte aufgeteilt, die miteinander interagieren. Ein Objekt fasst die \textbf{Attribute} eines Objekts und die \textbf{Methoden}, die damit arbeiten, zusammen.\par

In JavaScript und anderen Sprachen definiert man dafür \textbf{Klassen} und daraufhin kann man von diesen Klassen \textbf{Instanzen} erzeugen (instanziieren).

Klassen können von anderen Klassen \textbf{erben}, so können Klassen immer speziellere Probleme lösen.

Dieses Cheatsheet geht von grundlegenden JavaScript-Kenntnissen aus.

\subsection*{Klassendefinition}

Klassen sind die Vorlagen oder Blaupausen für Ihre späteren Objekte. Üblicherweise werden aus einer einzelnen Klasse viele Instanzen erzeugt. Zum Beispiel hat man eine \texttt{Person}-Klasse, aber viele Instanzen, die jeweils einzelne Personen repräsentieren.

\begin{codebox}{js}{}
class Person {
  constructor(name, occupation) {
    this.name = name;
    this.occupation = occupation;
  }

  greet() {
    console.log(this.name  + " says hi!");
  }
}
\end{codebox}

Eine Klasse definiert man mit dem Schlüsselwort \texttt{class} gefolgt vom Namen der Klasse. Der Name wird üblicherweise groß geschrieben. In den darauf folgenden geschweiften Klammern ist die definition der Klasse enthalten.

\textbf{Methoden} sind die Funktionen eines Objekts, innerhalb einer Klassendefinition brauchen sie aber kein \texttt{function}-Schlüsselwort. Man schreibt sie einfach direkt hin, wie \texttt{greet()} oder \texttt{constructor()} im vorigen Beispiel.

\subsection*{Instanz(en)}
\begin{codebox}{js}{}
let peter = new Person("Peter Parker", "photographer");
peter.greet();
// gibt "Peter Parker says hi!" aus.

let pepper = new Person("Virginia Potts", "secretary");
console.log(pepper.occupation);
// gibt "secretary" aus.
\end{codebox}

Instanzen erzeugen wir mit dem \texttt{new}-Schlüsselwort. Wir erhalten ein Objekt, dessen Methoden wir aufrufen können (\texttt{peter.greet();}) oder dessen Attribute wir abfragen und ändern können (\texttt{pepper.occupation}). Dabei verwenden wir die Schreibweise mit dem Punkt, die Sie auch aus einfachen JavaScript-Objekten schon kennen.

\subsection*{Konstruktor}
In den vorigen Beispielen haben wir den Konstruktor gleich schon verwendet. Dies ist eine besondere Methode, die immer gleich heißt (\texttt{constructor()}), die automatisch von JavaScript beim instanziieren aufgerufen wird. Die Parameter, die wir \texttt{new Person("Peter Parker", "photographer");} mitgeben, landen im \texttt{constructor()} und können dort verarbeitet -werden.

\subsection*{\texttt{this}}
Wir haben nun verschiedene Instanzen unserer Klasse. Innerhalb der Klasse können wir die Variablen \texttt{peter} oder \texttt{pepper} nicht verwenden, wir müssen also anders an "unsere" Attribute und Methoden heran kommen. Dafür ist das Schlüsselwort \texttt{this} da. \texttt{this} bezieht sich immer auf die gerade verwendete Instanz. Deswegen haben die beiden nachfolgenden Aufrufe verschiedene Ausgaben:

\begin{codebox}{js}{}
peter.greet();
// gibt "Peter Parker says hi!" aus.

pepper.greet();
// gibt "Virvinia Potts says hi!" aus.
\end{codebox}


\subsection*{Vererbung}
Klassen können voneinander erben. Dadurch kann man spezielle Klassen von allgemeinen Klassen ableiten, ohne dass man den allgemeinen Code kopieren muss.

\begin{codebox}{js}{}
class Shopkeeper extends Person {
  constructor(name) {
    super(name, "shopkeeper");

    this.items = [];
    this.money = 0;
  }

  buy(item, cost) {
    this.items.push(item);
    this.money -= cost;
  }

  sell(item, price) {
    const index = this.items.indexOf(item);
    if (index > -1) {
      this.items.splice(index, 1);
      this.money += cost;
    } else {
      console.error(`${this.name} does not have ${item}`);
    }
  }
}

let tom = new Shopkeeper("Tom Nook");
tom.buy("flower", 5);
tom.sell("flower", 6);
console.log(tom.money); // 1
tom.greet(); // "Tom Nook says hi!"
console.log(tom.occupation); // "shopkeeper"
\end{codebox}

\texttt{Shopkeeper} erbt in diesem Beispiel von \texttt{Person}. Daher können wir auf \texttt{occupation} oder \texttt{name} oder \texttt{greet()} zurück greifen ohne dass wir sie neu erstellen müssten. Gleichzeitig erweitert der Shopkeeper eigene Attribute (\texttt{items} und \texttt{money}) und Methoden (\texttt{buy()} und \texttt{sell()}).

\subsection*{\texttt{super}}
Im Zusammenhang mit Vererbung will man oftmals den Konstruktor der übergeordneten Klasse ebenfalls ausführen. Dafür gibt es die besondere Funktion \texttt{super()}. Im \texttt{Shopkeeper}-Beispiel kann man sehen, dass der neue Konstruktor auch andere Parameter übernehmen und weiter geben kann.

\subsection*{(todo, merkliste)}
Getter? Setter?

Klassendiagramm



%new page starts heres
\newpage

\subsection*{ES6 Module}
Objektorientierung und Module sind zwei verschiedene Dinge, aber sie gehen gerne Hand in Hand.

%\subsection*{Grundgerüst}
%Einige Dinge muss jedes HTML-Dokument enthalten, das Folgende ist das Minimale, was eine HTML-Datei braucht:

%Das \texttt{<html>}-Tag. (\enquote{Nachkommen} bzw. \enquote{Kind-Elemente}):

%% These are actually no longer neccessary, but this would be one way of leaving an intended gap at the bottom of a column:
% \vfill\null
% \columnbreak

%-------------MAIN CONTENT DONE----------------


\printbibliography
\attribution{Claudius Coenen}
\end{multicols}
\end{document}
